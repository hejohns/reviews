% NOTE: based off kaobook minimal_book example
%%%%%%%%%%
% Load the kaobook class
\documentclass[
	fontsize=10pt, % Base font size
	twoside=false, % Use different layouts for even and odd pages (in particular, if twoside=true, the margin column will be always on the outside)
	%open=any, % If twoside=true, uncomment this to force new chapters to start on any page, not only on right (odd) pages
	secnumdepth=-1, % How deep to number headings. Defaults to 1 (sections)
]{kaobook}

% Choose the language
\usepackage[english]{babel} % Load characters and hyphenation
\usepackage[english=american]{csquotes}	% English quotes

% Load packages for testing
\usepackage{blindtext}
%\usepackage{showframe} % Uncomment to show boxes around the text area, margin, header and footer
%\usepackage{showlabels} % Uncomment to output the content of \label commands to the document where they are used

% Load the bibliography package
\usepackage{kaobiblio}
\addbibresource{minimal.bib} % Bibliography file

% Load mathematical packages for theorems and related environments
\usepackage{kaotheorems}

% Load the package for hyperreferences
\usepackage{kaorefs}

\graphicspath{{images/}{./}} % Paths where images are looked for

\makeindex[columns=3, title=Alphabetical Index, intoc] % Make LaTeX produce the files required to compile the index


% my additional preamble stuff
\usepackage{hejohns-fonts} % my own font stuff
\usepackage{hejohns-colors}
\usepackage{attachfile2}
\NewDocumentCommand{\Index}{m}{\index{#1}\purple{#1}}
\pgfkeys{%
    /Entry/.is family,
    /Entry,
    % NOTE: default is different from .default
    default/.style = {
        tags = ,
        date = \today,
    },
    % can be anything since we always use default style
    tags/.initial,
    date/.initial,
}
\NewDocumentCommand{\Entry}{O{}m}{%
    \pgfkeys{/Entry, default, #1}%
    \section{#2}
    \textbf{Tags: }\blue{\pgfkeysvalueof{/Entry/tags}}

    \textbf{Date: }\pgfkeysvalueof{/Entry/date}

}
\NewDocumentCommand{\Paper}{O{}mmm}{%{year}{author}{title}
    \Entry[#1]{#2 - #3, \textit{#4}}
}
\NewDocumentCommand{\Book}{O{}mmm}{%{year}{author}{title}
    \Entry[#1]{#2 - #3, \textit{#4}}
}
\NewDocumentCommand{\Video}{O{}mmmm}{%{year}{speaker}{title/topic}{link}
    \Entry[#1]{#2 - #3, \textit{#4}}
    \textbf{Link: }\href{#5}{recording}

}
\NewDocumentCommand{\Dict}{O{}m}{\Entry[#1]{\Index{#2}}}%{term}
\NewDocumentCommand{\See}{}{\textbf{See: }}
\newtheoremstyle{break}% https://tex.stackexchange.com/a/37805
{}%
{}%
{}%
{}%
{\bfseries}%
{}%
{\newline}%
{}%
\theoremstyle{break}
\newtheorem*{thm}{Theorem} % do not number theorems
\newtheorem*{Def}{Definition} % do not number definitions
\newtheorem*{Ex}{Example} % do not number examples

\begin{document}

%----------------------------------------------------------------------------------------
%	BOOK INFORMATION
%----------------------------------------------------------------------------------------

\titlehead{some notes of mine}
\title[Template for the {\normalfont\texttt{kaobook}} Class]{%
hejohns' notes}
\author[hejohns]{hejohns}
\date{\today}
\publishers{made with kaobook \\ set in EB Garamond}

%----------------------------------------------------------------------------------------

\frontmatter % Denotes the start of the pre-document content, uses roman numerals

%----------------------------------------------------------------------------------------
%	COPYRIGHT PAGE
%----------------------------------------------------------------------------------------

\makeatletter
\uppertitleback{\@titlehead} % Header

\lowertitleback{
	\textbf{Disclaimer} \\
	You can edit this page to suit your needs. For instance, here we have a no copyright statement, a colophon and some other information. This page is based on the corresponding page of Ken Arroyo Ohori's thesis, with minimal changes.
	
	\medskip
	
	\textbf{No copyright} \\
	\cczero\ This book is released into the public domain using the CC0 code. To the extent possible under law, I waive all copyright and related or neighbouring rights to this work.
	
	To view a copy of the CC0 code, visit: \\\url{http://creativecommons.org/publicdomain/zero/1.0/}
	
	\medskip
	
	\textbf{Colophon} \\
	This document was typeset with the help of \href{https://sourceforge.net/projects/koma-script/}{\KOMAScript} and \href{https://www.latex-project.org/}{\LaTeX} using the \href{https://github.com/fmarotta/kaobook/}{kaobook} class.
	
	\medskip
	
	\textbf{Publisher} \\
	First printed in May 2019 by \@publishers
}
\makeatother

%----------------------------------------------------------------------------------------
%	DEDICATION
%----------------------------------------------------------------------------------------

% NOTE: keeping this here to preserve minimal_book formatting
%\dedication{
%	The harmony of the world is made manifest in Form and Number, and the heart and soul and all the poetry of Natural Philosophy are embodied in the concept of mathematical beauty.\\
%	\flushright -- D'Arcy Wentworth Thompson
%}

%----------------------------------------------------------------------------------------
%	OUTPUT TITLE PAGE AND PREVIOUS
%----------------------------------------------------------------------------------------

% Note that \maketitle outputs the pages before here
\maketitle

%----------------------------------------------------------------------------------------
%	PREFACE
%----------------------------------------------------------------------------------------

%\chapter*{Preface}
%
%\blindtext

%----------------------------------------------------------------------------------------
%	TABLE OF CONTENTS & LIST OF FIGURES/TABLES
%----------------------------------------------------------------------------------------

\begingroup % Local scope for the following commands

% Define the style for the TOC, LOF, and LOT
%\setstretch{1} % Uncomment to modify line spacing in the ToC
%\hypersetup{linkcolor=blue} % Uncomment to set the colour of links in the ToC
\setlength{\textheight}{230\vscale} % Manually adjust the height of the ToC pages

% Turn on compatibility mode for the etoc package
\etocstandarddisplaystyle % "toc display" as if etoc was not loaded
\etocstandardlines % "toc lines as if etoc was not loaded

\tableofcontents % Output the table of contents

\listoffigures % Output the list of figures

% Comment both of the following lines to have the LOF and the LOT on different pages
\let\cleardoublepage\bigskip
\let\clearpage\bigskip

\listoftables % Output the list of tables

\endgroup

%----------------------------------------------------------------------------------------
%	MAIN BODY
%----------------------------------------------------------------------------------------

\mainmatter % Denotes the start of the main document content, resets page numbering and uses arabic numbers
\setchapterstyle{kao} % Choose the default chapter heading style

\chapter{Papers}
\Paper[tags={computability,history},date=2022-12-17]{2012}{Soare}{Formalism and intuition in computability}
up to p.9

f
\Paper[tags={logic,computability,history}]{1981}{Kleene}{Origins of recursive function theory}
    need to read last ~10 pages

    nice review by Steward Shapiro, 1990

    λ-defineable = Church,
    recursive = Gödel, Herbrand,
    computable = Turing,

    (although until 193?, recursive $↦$ primitive recursive, for Gödel,
    now recursive $↦$ Herbrand-Gödel general recursive)
\chapter{Book Reviews}
\Book[]{1995}{Makiko Nakano}{Makiko's Diary}
    Translated by Kazuko Smith

    sticky rice (desserts) = ½sticky rice + ½normal
\chapter{Presentations/Lectures}
\Video[]{2007}{Bryan Cantrill}{Dtrace}{https://www.youtube.com/watch?v=6chLw2aodYQ}
    A
\chapter{Dictionary}
\Dict[tags={logic,computability},date=2022-12-18]{Craig's Trick}
    \begin{thm}[Craig's Trick]
        \marginnote{%
            So ce theories are just as effective as computable ones,
            which is good news for axiom schemas.
            A priori, it's not clear that theories w/ axiom schemas are as effective
            as finitely axiomatizable ones,
            but intuitively, axiom schemas are of the same character,
            are ``easily checkable".
            Craig's trick formally grounds this.
        }
        \marginnote{%
            eg $PA$ is as effective as $PA^-$.
            (although $PA^-$ has nice utility for the Entscheidungsproblem.)
        }
        From Mathew (MATH 684):
        $S$ ce set of sentences $⟹ ∃S^*.S^*$ computable $∧$ they have the same theory.

        (my terminology) a theory is computable $⟺$ it is ce
        \sidenote{…sentences up to logical equivalence}

        ie a theory is computably axiomatized $⟺$ it is computably enumerably axiomatized
    \end{thm}
    \begin{proof}
        \marginnote{%
            Wikipedia has a similar sketch.
        }
        MATH 684:
        $S$ ce, so you only have a listing of the sentences.
        We can make it strictly monotonic by relisting,
        but by adding a bunch of tautological noise or padding to each sentence
        (say, by conjuncting tautologies, assuming eg Gödel's prime factorization encoding)
        st the Gödel number is much larger.
        (Each sentence is relisted logically equivalently.)
        \marginnote{%
            Somehow the proof itself doesn't feel intuitive,
            but the ``intended use" of the theorem \emph{is}.
        }
    \end{proof}

    \See Theory
\Dict[tags={logic,computability},date=2022-12-18]{Enumeration Operator}
    \marginnote{From Rogers' 1967 \textit{Theory of Recursive Functions}.}

    \begin{Def}[Enumeration Operator]
        \marginnote{%
            An archetypical example:
            In Gödel's incompleteness theorems,
            each computable enumeration of a theory (my sense) gives rise to
            an enumeration of the theory (the deductive closure).
            In Miller's terms,
            there is the ``deducibility operator" $D$ that gives for each axiom set $B$,
            $D(B)$, the set of consequences.
        }
        Each enumeration reduction witness $z$ and $B$ determine the $A$,
        so each $z$ determines a enumeration operator $Φ_z : 2^ω → 2^ω$.

        ie $Φ_z(B) = A ⟺ A ≤_e X$ witnessed by $z$.

        $A ≡_e B ⟺ A ≤_e B ∧ B ≤_e A$
    \end{Def}
    \begin{thm}
        \marginnote{I'm thinking of the deducibility operator the whole time.}
        \begin{itemize}
            \item[]
            \item enumeration operators compose, by inspection
            \item $A ⊆ B ⟹ Φ(A) ⊆ Φ(B)$ (monotonicity)
            \item $x ∈ Φ(B) ⟹ ∃C.C \text{ finite } ∧ C ⊆ B ∧ x ∈ Φ(C)$ (continuity)
            \sidenote{which I'd call compactness}
        \end{itemize}
    \end{thm}

    \See Dana Scott's graph model of λ-calculus
\Dict[tags={logic,computability},date=2022-12-18]{Enumeration Reducibility}
    \marginnote[-2.7cm]{From Rogers' 1967 \textit{Theory of Recursive Functions}.}
    \begin{Def}[Enumeration Reduc(tion/ible)]
        \marginnote[-2.7cm]{%
            This definition is not as nice as (many-)one or Turing reduction,
            but the idea is that we want to
            ``effectively list $A$ using any listing (computable or not) of $B$".
            Note that enumeration reductions ``only use positive information about $B$,
            and produce only positive information about $A$;
            whereas Turing reductions use and produce both positive and negative information."
            (paraphrased from the introduction to Russell Miller's \textit{Non-coding Enumeration Operators}.)
        }
        $A ≤_e B ⟺ ∃z.∀x.x ∈ A ↔ ∃u.⟨x,y⟩ ∈ W_z ∧ D_u ⊆ B$

        where $z$ is the Gödel code of the reduction witness,
        and $D_u$ is the finite set associated with $u$ as a canonical index (ie a tuple).
        \sidenote{%
            The idea w/ $u$ is that to list $A$ while watching elements enter $B$,
            you should only need (to see) a finite amount of $B$ to list a particular element $x ∈ A$.
        }
    \end{Def}
    Rogers' (really simple) examples:
    \begin{itemize}
        \item $\{2n | n ∈ ω\} ≤_e ω$
        \item $A$ ce $⟹$ $∀B.A ≤_e B$
    \end{itemize}

    \See Enumeration Operator
\Dict[tags={logic,λ-calculus},date=2022-12-18]{Dana Scott's Graph Model}
    \begin{Def}
        \begin{align}
            ⟦λx.⟧ &:=  \\
            ⟦e_1 e_2⟧ &:= 
        \end{align}
    \end{Def}

    \See Enumeration Operator
\Dict[tags={logic},date=2022-12-17]{Herbrand's Theorem}
    TODO: convert notes from Prof. Blass' November seminar
    \Index{Herbrand's Theorem}
\Dict[tags={logic},date=2022-12-19]{Knaster-Tarski}
    \begin{thm}[Knaster-Tarski Fixpoint Theorem]
        \marginnote{%
            This theorem has many statements, and this is the easiest for me to remember.
            The complete lattice is often a powerset lattice.
        }
        Every monotone function on a complete lattice has a complete lattice of fixpoints.
    \end{thm}
    \begin{proof}
        Widely available.
    \end{proof}
    \begin{Ex}
        \begin{itemize}
            \item[]
            \item The deducibility operator is a monotone function on sets of sentences.
                \sidenote{For simplicity, assume everything is about-- and still true about-- a fixed language of arithmetic.}
                The bottom (least) fixpoint is the (deductively closed) empty theory.
                The top (greatest) fixpoint is the inconsistent theory, ie the set of all sentences.
                Consistency of the empty theory
                (by Gentzen's original cut elimination, or by existence of a model)
                says this complete lattice is nontrivial.

                Any consistent, computably axiomatizable (deductively closed) theory that proves more than the empty theory
                is an intermediate fixpoint-- eg PA.
                Incompleteness says there is no intermediate fixpoint above PA that is complete,
                but there are at least $2^{ℵ_0}$ intermediate fixpoints above PA where we keep adding $Con(T)$ or $¬Con(T)$.
                \sidenote{Are there complete intermediate fixpoints?}
        \end{itemize}
    \end{Ex}

    \See Enumeration Operator, Dana Scott's Graph Model
\Dict{Locus Solum}
    This is my version of Girard's dictionary.

    Also afaik the most Girard paper there is
\Dict{$PA$}
    Peano Arithmetic w/ induction.
    $PA^-$ := PA ∖ induction
\Dict{Realizability}
    This is how we can attach beamer presentations
    \attachfile{../../22f/rg/pfenning-davies/slides.pdf}
\Dict[tags={logic},date=2022-12-18]{Theory}
    An unfortunately ambiguous term,
    but you can usually figure it out from context,
    if it really matters.

    I tend to use ``Theory" to just mean a set of sentences,
    as in the $Γ$ in the sequent $Γ ⊢ $.
    So I see a finite set for ``the theory of groups",
    and a finite set unioned w/ a schema for ``Peano Arithmetic".
    (and the empty set for the Entscheidungsproblem.)
    Sometimes, people mean a deductively closed set of sentences.

%\pagelayout{wide} % No margins
%\addpart{Title of the Part}
%\pagelayout{margin} % Restore margins
%
%
%\appendix % From here onwards, chapters are numbered with letters, as is the appendix convention
%
%\pagelayout{wide} % No margins
%\addpart{Appendix}
%\pagelayout{margin} % Restore margins
%
%\chapter{Some more blindtext}
%
%\blindtext

%----------------------------------------------------------------------------------------

\backmatter % Denotes the end of the main document content
\setchapterstyle{plain} % Output plain chapters from this point onwards

%----------------------------------------------------------------------------------------
%	BIBLIOGRAPHY
%----------------------------------------------------------------------------------------

% The bibliography needs to be compiled with biber using your LaTeX editor, or on the command line with 'biber main' from the template directory

\defbibnote{bibnote}{Here are the references in citation order.\par\bigskip} % Prepend this text to the bibliography
\printbibliography[heading=bibintoc, title=Bibliography, prenote=bibnote] % Add the bibliography heading to the ToC, set the title of the bibliography and output the bibliography note

%----------------------------------------------------------------------------------------
%	INDEX
%----------------------------------------------------------------------------------------

% The index needs to be compiled on the command line with 'makeindex main' from the template directory

\printindex % Output the index

\end{document}
