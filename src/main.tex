% NOTE: based off kaobook minimal_book example
%%%%%%%%%%
% Load the kaobook class
\documentclass[
	fontsize=10pt, % Base font size
	twoside=false, % Use different layouts for even and odd pages (in particular, if twoside=true, the margin column will be always on the outside)
	%open=any, % If twoside=true, uncomment this to force new chapters to start on any page, not only on right (odd) pages
	secnumdepth=-1, % How deep to number headings. Defaults to 1 (sections)
]{kaobook}

% Choose the language
\usepackage[english]{babel} % Load characters and hyphenation
\usepackage[english=american]{csquotes}	% English quotes

% Load packages for testing
\usepackage{blindtext}
%\usepackage{showframe} % Uncomment to show boxes around the text area, margin, header and footer
%\usepackage{showlabels} % Uncomment to output the content of \label commands to the document where they are used

% Load the bibliography package
\usepackage{kaobiblio}
\addbibresource{minimal.bib} % Bibliography file

% Load mathematical packages for theorems and related environments
\usepackage{kaotheorems}

% Load the package for hyperreferences
\usepackage{kaorefs}

\graphicspath{{images/}{./}} % Paths where images are looked for

\makeindex[columns=3, title=Alphabetical Index, intoc] % Make LaTeX produce the files required to compile the index


% my additional preamble stuff
\usepackage{hejohns-fonts} % my own font stuff
\usepackage{hejohns-colors}
\usepackage{attachfile2}
\NewDocumentCommand{\Index}{m}{\index{#1}\purple{#1}}
\pgfkeys{%
    /Entry/.is family,
    /Entry,
    default/.style = {
        tags = \today,
        date = \today,
    },
    tags/.initial = undef,
    date/.initial = undef,
}
\NewDocumentCommand{\Entry}{O{}m}{%
    \pgfkeys{/Entry, default, #1}%
    \section{#2}
    \textbf{Tags: }\blue{\pgfkeysvalueof{/Entry/tags}}

    \textbf{Date: }\pgfkeysvalueof{/Entry/Date}

}
\NewDocumentCommand{\Paper}{O{}mmm}{%{year}{author}{title}
    \Entry[#1]{#2 - #3, \textit{#4}}
}

\begin{document}

%----------------------------------------------------------------------------------------
%	BOOK INFORMATION
%----------------------------------------------------------------------------------------

\titlehead{some notes of mine}
\title[Template for the {\normalfont\texttt{kaobook}} Class]{%
hejohns' notes}
\author[hejohns]{hejohns}
\date{\today}
\publishers{made with kaobook, set in EB Garamond}

%----------------------------------------------------------------------------------------

\frontmatter % Denotes the start of the pre-document content, uses roman numerals

%----------------------------------------------------------------------------------------
%	COPYRIGHT PAGE
%----------------------------------------------------------------------------------------

\makeatletter
\uppertitleback{\@titlehead} % Header

\lowertitleback{
	\textbf{Disclaimer} \\
	You can edit this page to suit your needs. For instance, here we have a no copyright statement, a colophon and some other information. This page is based on the corresponding page of Ken Arroyo Ohori's thesis, with minimal changes.
	
	\medskip
	
	\textbf{No copyright} \\
	\cczero\ This book is released into the public domain using the CC0 code. To the extent possible under law, I waive all copyright and related or neighbouring rights to this work.
	
	To view a copy of the CC0 code, visit: \\\url{http://creativecommons.org/publicdomain/zero/1.0/}
	
	\medskip
	
	\textbf{Colophon} \\
	This document was typeset with the help of \href{https://sourceforge.net/projects/koma-script/}{\KOMAScript} and \href{https://www.latex-project.org/}{\LaTeX} using the \href{https://github.com/fmarotta/kaobook/}{kaobook} class.
	
	\medskip
	
	\textbf{Publisher} \\
	First printed in May 2019 by \@publishers
}
\makeatother

%----------------------------------------------------------------------------------------
%	DEDICATION
%----------------------------------------------------------------------------------------

% NOTE: keeping this here to preserve minimal_book formatting
%\dedication{
%	The harmony of the world is made manifest in Form and Number, and the heart and soul and all the poetry of Natural Philosophy are embodied in the concept of mathematical beauty.\\
%	\flushright -- D'Arcy Wentworth Thompson
%}

%----------------------------------------------------------------------------------------
%	OUTPUT TITLE PAGE AND PREVIOUS
%----------------------------------------------------------------------------------------

% Note that \maketitle outputs the pages before here
\maketitle

%----------------------------------------------------------------------------------------
%	PREFACE
%----------------------------------------------------------------------------------------

%\chapter*{Preface}
%
%\blindtext

%----------------------------------------------------------------------------------------
%	TABLE OF CONTENTS & LIST OF FIGURES/TABLES
%----------------------------------------------------------------------------------------

\begingroup % Local scope for the following commands

% Define the style for the TOC, LOF, and LOT
%\setstretch{1} % Uncomment to modify line spacing in the ToC
%\hypersetup{linkcolor=blue} % Uncomment to set the colour of links in the ToC
\setlength{\textheight}{230\vscale} % Manually adjust the height of the ToC pages

% Turn on compatibility mode for the etoc package
\etocstandarddisplaystyle % "toc display" as if etoc was not loaded
\etocstandardlines % "toc lines as if etoc was not loaded

\tableofcontents % Output the table of contents

\listoffigures % Output the list of figures

% Comment both of the following lines to have the LOF and the LOT on different pages
\let\cleardoublepage\bigskip
\let\clearpage\bigskip

\listoftables % Output the list of tables

\endgroup

%----------------------------------------------------------------------------------------
%	MAIN BODY
%----------------------------------------------------------------------------------------

\mainmatter % Denotes the start of the main document content, resets page numbering and uses arabic numbers
\setchapterstyle{kao} % Choose the default chapter heading style

\chapter{Papers}
\Paper[tags={computability,history},date=2022-12-17]{2012}{Soare}{Formalism and intuition in computability}
up to p.9

f
\Paper[tags={logic,computability,history}]{1981}{Kleene}{Origins of recursive function theory}
    need to read last ~10 pages

    nice review by Steward Shapiro, 1990

    λ-defineable = Church,
    recursive = Gödel, Herbrand,
    computable = Turing,

    (although until 193?, recursive $↦$ primitive recursive, for Gödel,
    now recursive $↦$ Herbrand-Gödel general recursive)
\chapter{Dictionary}
\Entry[tags={logic,traditional},date=2022-12-17]{Herbrand's Theorem}
    TODO: convert notes from Prof. Blass' November seminar
    \Index{Herbrand's Theorem}
\Entry{Locus Solum}
This is my version of Girard's dictionary.

Also afaik the most Girard paper out there is
\Entry{Realizability}
This is how we can attach beamer presentations
\attachfile{../../22f/rg/pfenning-davies/slides.pdf}

%\pagelayout{wide} % No margins
%\addpart{Title of the Part}
%\pagelayout{margin} % Restore margins
%
%
%\appendix % From here onwards, chapters are numbered with letters, as is the appendix convention
%
%\pagelayout{wide} % No margins
%\addpart{Appendix}
%\pagelayout{margin} % Restore margins
%
%\chapter{Some more blindtext}
%
%\blindtext

%----------------------------------------------------------------------------------------

\backmatter % Denotes the end of the main document content
\setchapterstyle{plain} % Output plain chapters from this point onwards

%----------------------------------------------------------------------------------------
%	BIBLIOGRAPHY
%----------------------------------------------------------------------------------------

% The bibliography needs to be compiled with biber using your LaTeX editor, or on the command line with 'biber main' from the template directory

\defbibnote{bibnote}{Here are the references in citation order.\par\bigskip} % Prepend this text to the bibliography
\printbibliography[heading=bibintoc, title=Bibliography, prenote=bibnote] % Add the bibliography heading to the ToC, set the title of the bibliography and output the bibliography note

%----------------------------------------------------------------------------------------
%	INDEX
%----------------------------------------------------------------------------------------

% The index needs to be compiled on the command line with 'makeindex main' from the template directory

\printindex % Output the index

\end{document}
